The recipes in this book are provided with the sourdough option by default.

For every recipe fresh or dry yeast can be used interchangeably. If the recipe
calls for around 10\% of sourdough starter you can replace this with roughly
0.1\% of dry yeast or 0.3\% of fresh yeast.

If you are in a rush some of the recipes can also be made with more yeast.
This can sometimes make sense if you want to opt for a fast leavened dough
such as burger buns or naans for instance. You can increase the dry yeast
percentage to around 1\% of the flour. If you want to opt for fresh yeast
multiply that with 3. That would be an equivalent of 3\% of fresh yeast in
terms of baker's math. When opting for a faster leavened dough it is
recommended to add some active malt. Opt for around 5\% based on the flour.
The malt will improve the enzymatic activity of your dough and provide better
results when making a faster dough.

Please note though that while you can speed up the process, this also
reduces the quality of the final bread. A slow fermentation is the key
to making great bread at home. However - sometimes you might be in a rush
and you can increase the fermentation times with the aforementioned tips.
Every homemade bread is better than store-bought bread.